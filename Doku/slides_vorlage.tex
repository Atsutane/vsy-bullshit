\documentclass{beamer}
%\documentclass[draft]{beamer} %Zum schnelleren Kompilieren beim Entwickeln
%\documentclass[notes=show]{beamer}
%\usepackage[latin1]{inputenc}
\usepackage[utf8]{inputenc}
\usepackage{ngerman}
\usepackage{rotating}
\usepackage{verbatim}
\usepackage{latexsym}
\usepackage{color}
\usepackage{graphicx}
\usepackage{tabularx}
\usepackage{ragged2e}
\usepackage{eurosym}
\selectlanguage{german}

%\definecolor{lightblue}{rgb}{0.8,0.9,1.0}
%\usetheme{PaloAlto}
%\usetheme{Darmstadt}
\usetheme{Frankfurt}

% LaTeX-interne Einstellungen zum Umbruch etc.
% \nofiles    % Inhaltsverzeichnis nicht ändern.
\hbadness 10000
\sloppy
\frenchspacing

%\usecolortheme{crane}
%\usecolortheme{lily}

\newcolumntype{Y}{>{\centering\arraybackslash}X}%
\newcolumntype{Z}{>{\hfill\arraybackslash}X}%

\usepackage{amsmath,amssymb}

\setbeamercovered{dynamic}

%notes:
\setbeamertemplate{note page}[plain] 
%compressed

\newcommand{\datum}{1.3.2011}
\newcommand{\autor}{Christian Baun}
\newcommand{\vorlesung}{Verteilte Systeme}
\newcommand{\semester}{SS2011}
\newcommand{\ort}{Hochschule Mannheim}

\title[]{Verteilte Systeme\\ \ort}
\author{\autor}
\institute{Fakultät für Informatik\\
           Hochschule Mannheim\\
          \texttt{cray@unix-ag.uni-kl.de}}
\date{\datum}
\beamertemplatetransparentcovereddynamic
\setbeamerfont{note page}{size=\small}
%\renewcommand{\>}{\rangle}
%\newcommand{\<}{\langle} 
%\usebeamercolor[fg]{[page number]}

\setbeamertemplate{navigation symbols}{}
%\setbeamertemplate{note page}{}

%Zum Abschalten der kleinen Navigationsleiste am unteren Rand reicht folgende Zeile aus:
%\beamertemplatenavigationsymbolsempty

\setbeamersize{text margin left=0.2cm}
\setbeamersize{text margin right=0.25cm}
\setbeamersize{sidebar width right=0cm}
\setbeamersize{sidebar width left=0cm}

%Inhalt der Fußzeile festlegen
\setbeamertemplate{footline}
{
\begin{beamercolorbox}{title}
\hspace*{0.2cm}
% \copyright 
\autor\ -- \vorlesung\ -- \ort\ -- \semester
\hfill
%  \insertslidenavigationsymbol
%  \insertframenavigationsymbol
%  \insertsubsectionnavigationsymbol
%  \insertsectionnavigationsymbol
%  \insertdocnavigationsymbol
%  \insertbackfindforwardnavigationsymbol
 \hfill\insertframenumber/\inserttotalframenumber
 \hspace*{0.2cm}
\end{beamercolorbox}
}

%\newcommand{\markNote}[1]{	\textsuperscript{\tiny [#1]} }
\newcommand{\markNote}[1]{	 }

% Bei jeder Section ein Inhaltsverzeichnis ausgeben
% \AtBeginSection[]{
% \begin{frame}
% \frametitle{Outline}
% \tableofcontents[current,hideallsubsections] % Nur die sections ausgeben
% \end{frame}
% }

% Die Zeile definieren, die ganz oben ist und die Abschnitte enthält
% Das sind die Maße wenn es die Folie "nächste Vorlesung gibt"
% \setbeamertemplate{headline}
% {%
%   \begin{beamercolorbox}{section in head/foot}
%     \vskip5pt\insertnavigation{12.4cm}\vskip5pt
%   \end{beamercolorbox}%
% }

\setbeamertemplate{headline}
{%
  \begin{beamercolorbox}{section in head/foot}
    \vskip5pt\insertnavigation{12.75cm}\vskip5pt
  \end{beamercolorbox}%
}

\begin{document}

% Deckblatt generieren
\begin{frame}
\titlepage
\end{frame}

% Inhaltsverzeichnis am Anfang des Dokuments
% \begin{frame}
%   \frametitle{Heute}
%   %\tableofcontents[hideallsubsections]  % Nur die sections ausgeben
%   \tableofcontents                       % sections und subsections ausgeben
% \end{frame}


\section{Grundlagen}

\subsection{README}

\begin{frame}[fragile]
% [fragile] braucht man, wenn man \verb oder \verbatim auf der Folie verwendet!
\frametitle{README}
\begin{itemize}
\item Das hier ist die Folienvorlage für die praktische Übung der Vorlesung Verteilte Systeme im Sommersemester 2011
\item Diese Vorlage soll den Einstieg in \LaTeX\ und speziell Bib\TeX\ erleichtern
\item Wird unter Linux im Verzeichnis mit der Quelldatei (\texttt{.tex}) das Kommando \texttt{make} eingegeben, erzeugt \verb!pdflatex! eine PDF-Datei
\item Ein aktuelles \LaTeX\ (z.B. TeX Live\footnote{\texttt{http://www.tug.org/texlive/}}) sollte installiert sein
\item Wer unter Windows die Folien machen möchte, dem empfehle ich MiK\TeX\footnote{\texttt{http://www.miktex.org}} und \TeX nicCenter\footnote{\texttt{http://www.texniccenter.org}}
\item Ein leistungsfähiger Editor für \LaTeX\ unter MacOS X ist TeXShop\footnote{\texttt{http://pages.uoregon.edu/koch/texshop/}}
\end{itemize}
\end{frame}

\begin{frame}[fragile]
\frametitle{\LaTeX Beamer}
\begin{itemize}
\item Diese Vorlage nutzt die \LaTeX-Klasse \texttt{beamer}.
\item Gute Dokumentation zur Klasse \texttt{beamer}:
\verb!http://www2.informatik.hu-berlin.de/~mischulz/beamer.html!
\end{itemize}
\end{frame}

\section{Erfolgreich präsentieren}
\begin{frame}
\frametitle{Medieneinsatz}
\begin{itemize}
\item Auf den folgenden Folien kommen ein paar Ratschläge zum erfolgreichen Präsentieren
\item Ablauf einer Präsentation
\begin{itemize}
\item Einleitung mit Agenda
\begin{itemize}
\item Vorstellung des Referenten und des Themas
\item Aufmerksamkeit beim Auditorium erwecken
\end{itemize}
\item Hauptteil
\begin{itemize}
\item An die in der Inhaltsangabe aufgeführten
\item Roter Faden muss erkennbar sein
\end{itemize}
\item Schluss mit Zusammenfassung und Fazit/Resümee
\begin{itemize}
\item Gesamtaussage des Vortrages noch einmal kurz wiederholen
\item Wichtige Erkenntnisse zusammenfassen
\item Für die Aufmerksamkeit danken und Fragen zulassen
\end{itemize}
\end{itemize}
\end{itemize}
\end{frame}


\subsection{Gute Folien machen}
\begin{frame}
\frametitle{Gute Folien machen}
\begin{itemize}
\item Serifenlose Schriften verwenden. Diese sind besser (leichter) lesbar
\item Nicht zu viel Text auf den Folien
\item Text durch Punkte und Kästen gliedern
\item Sätze sollten nicht zu lang sein
\item Ein Bild sagt mehr als 1000 Worte
\item Durchgängig Großschreibung am Anfang oder durchgängig darauf verzichten
\item Entweder immer ein Punkt am Ende oder nie
\item Rechtschreibung durch geeignete Software und Freunde/Verwandte kontrollieren lassen
\end{itemize}
\end{frame}

\subsection{Dauer einer Präsentation}
\begin{frame}
\frametitle{Dauer einer Präsentation}
\begin{itemize}
\item Versuchen Sie auf jeden Fall die Gesamtzeitvorgaben einzuhalten
\item Überziehen gilt als Unhöflichkeit gegenüber dem Auditorium weil dadurch eventuell die  vorgesehene Diskussionszeit beschnitten oder bestehende Terminplanungen beeinflusst werden
\item Eine Ausnahme ist, wenn während der Präsentation Fragen entstehen
\begin{itemize}
\item Wichtige Fragen während der Präsentation sollten sie zulassen
\item Eventuelle, daraus resultierende Verzögerungen sind dem Vortragenden nicht anzulasten
\end{itemize}
\end{itemize}
\end{frame}

\subsection{Vortragstechnik}
\begin{frame}
\frametitle{Vortragstechnik}
\begin{itemize}
\item Klar und deutlich zum Auditorium sprechen
\item Möglichst frei sprechen
\item Auf aufgelegte Folien eingehen ohne diese 1:1 herunterzulesen
\item Nicht mit dem Rücken zum Auditorium vortragen
\item Das Auditorium motivieren, indem begründet wird, warum das gesagte wichtig ist
\item Das Auditorium nicht durch zu monotonen Redefluss einschläfern
\item Aufmerksamkeit von Unruhestiftern kann durch Blickkontakt zurückgewonnen werden
\end{itemize}
\end{frame}



\subsection{Medieneinsatz}
\begin{frame}
\frametitle{Medieneinsatz}
\begin{itemize}
\item Ein paar Worte zum Medieneinsatz in Präsentationen
\item Beim Zuhörer wird durch das gleichzeitige Sehen und Hören eine bessere Wirkung erzielt
\item Es existieren Erfahrungswerte, wie stark die Aufnahme von Informationen von der entsprechenden Darbietung abhängig ist
\item Demnach behält ein Mensch ca.
\begin{itemize}
\item 20\% von dem was er hört
\item 30\% von dem was er sieht
\item 50\% von dem was er hört und sieht
\end{itemize}
\item Aufgelegte Folien sollten nicht zu schnell wechseln
\item Das Auditorium muss die Folien leicht nachvollziehen und mitlesen können
\end{itemize}
\end{frame}

\subsection{Anglizismen}
\begin{frame}
\frametitle{Anglizismen (Deppenapostroph)}
\begin{itemize}
\item Anglizismengebrauch ist häufig nichts weiter als Angeberei oder Imponiergehabe
\item Vermeiden sie das Deppenapostroph
\begin{itemize}
\item Dabei wird das \emph{s} im Genitiv oder Plural durch einen Apostroph abgetrennt, was  grammatikalisch völlig falsch ist
\item Beispiele: \glqq CD's\grqq\ oder \glqq Microsoft's neue Software\grqq
\end{itemize}
\item Passende Literatur zum Thema Anglizismen: \emph{Speak German!: Warum Deutsch manchmal besser ist.} Wolf Schneider. Rowohlt (2008)
\end{itemize}
\end{frame}


\begin{frame}
\frametitle{Anglizismen (Beispiele)}
\begin{itemize}
\item Verwenden sie wo möglich etablierte Wörter der deutschen Sprache
\end{itemize}

\begin{center}
\begin{tabular}{|l|l|}
\hline
Speicher             & Storage     \\\hline
Hauptspeicher        & Memory      \\\hline
Bildlaufleiste       & Scrollbar   \\\hline
Knopf                & Button      \\\hline
Leistung/Durchsatz   & Performance \\\hline
Folien               & Slides      \\\hline
Vortrag              & Talk        \\\hline
Treffen              & Meeting     \\\hline
Pause                & Break       \\\hline
Bildschirm           & Display     \\\hline
Netzauftritt         & Website     \\\hline
Mobiltelefon         & Handy\footnote{Bedeutung ist im Englischen ganz anders. Korrekt wäre \emph{mobile phone} oder \emph{cell phone}} \\\hline
\end{tabular}
\end{center}
\end{frame}

\subsection{Bindestrich}
\begin{frame}
\frametitle{Bindestrich}
\begin{itemize}
\item In der deutschen Sprache werden Bindestriche nur sparsam eingesetzt
\item Werden in einem deutschen Text deutsche und englische Wörter verknüpft, ist ein Bindestrich angebracht
\begin{itemize}
\item Beispiele: Web-Anwendung, Cloud-Dienst, PDF-Datei, UNIX-Derivat, Scheduling-Lösung, EC2-Instanz
\end{itemize}
\item Bei englischen Begriffen, die aus zwei Wörtern bestehen, kann auf den Bindestrich verzichtet werden
\begin{itemize}
\item Beispiele: Cloud Computing, Grid Computing, Web Service, Open Source, Xen Hypervisor
\end{itemize}
\end{itemize}
\end{frame}

\section{Textformatierung}

\subsection{Schriften und Sonderzeichen}

\begin{frame}[fragile]
\frametitle{Schriften und Sonderzeichen}
\begin{itemize}
\item Es gibt verschiedene Schriftsätze: \textbf{Bild Face}, \textrm{Roman}, \textit{Italic}, \texttt{Typewriter}, \textsf{Sans Serif}, \textsl{Slanted}, \textsc{Small Caps}
\item \textcolor{blue}{Farben} sollte man \textcolor{red}{sparsam} einsetzen
\item Ein paar Sonderzeichen: \textbackslash, \$, \&, \euro, \%, \#, \textunderscore, \textasciitilde, \textasciicircum, \textbar, \{, \}
\item Weitere Sonderzeichen: \copyright, \textregistered, \texttrademark, \S, \P, \pounds, \dag, \ddag, \textbullet
\item Fortsetzungspunkte macht das Kommando \verb!\dots!. Ergebnis: \dots
\end{itemize}
\end{frame}

\subsection{Punkte}
\begin{frame}
\frametitle{Punkte}
\begin{itemize}
\item Punkt 1
\item Punkt 2
\begin{itemize}
\item Unterpunkt 1
\item Unterpunkt 2
\begin{itemize}
\item Unterunterpunkt 1
\item Unterunterpunkt 2
\item Unterunterpunkt 3
\item Unterunterpunkt 4
\end{itemize}
\item Unterpunkt 3
\end{itemize}
\item Punkt 3
\item Punkt 4
\end{itemize}
\end{frame}


\subsection{Blöcke}

\begin{frame}[t]
% [t] legt fest, dass der Text oben auf der Folie platziert wird.
\frametitle{Blöcke}
\begin{itemize}
\item Es gibt verschiedene Arten von Blöcken:
\end{itemize}
\begin{block}{Blocktitel}
Blocktext
\end{block}

\begin{exampleblock}{Blocktitel}
Blocktext
\end{exampleblock}

\begin{alertblock}{Blocktitel}
Blocktext
\end{alertblock}
\end{frame}

\subsection{Bilder}

\begin{frame}
% [t] legt fest, dass der Text oben auf der Folie platziert wird.
\frametitle{Bilder}
\begin{itemize}
\item Hier ist ein Bild:
\end{itemize}
\begin{center}
% Die Dateiendung muss man nicht angeben.
% Man hätte auch alternativ die Breite mit "width" angeben können.
\end{center}
\begin{itemize}
\item Bilder sollten im Dateiformat (\texttt{.pdf}) vorliegen. Dieses Dateiformat kann mit GIMP und vielen anderen Programmen erzeugt werden
\end{itemize}
\end{frame}

\subsection{Tabellen}

\begin{frame}[fragile]
\frametitle{Tabellen}
\begin{itemize}
\item Es gibt mehrere Umgebungen, um Tabellen zu setzen. \texttt{tabular} ist nur eine von vielen
\end{itemize}

\begin{center}
\begin{tabular}{|c|l|c|r|}
\hline
  \textbf{Zeile} & \textbf{Linksbündig} & \textbf{Zentriert} & \textbf{Rechtsbündig} \\
\hline\hline
  1 & Zeile 1 & Zeile 1 & Zeile 1 \\
  2 & Zeile 2 & Zeile 2 & Zeile 2 \\
  3 & Zeile 3 & Zeile 3 & Zeile 3 \\
\hline
\end{tabular}
\end{center}

\begin{itemize}
\item Das geht auch ohne Rahmen
\end{itemize}

\begin{center}
\begin{tabular}{clcr}
  \textbf{Zeile} & \textbf{Linksbündig} & \textbf{Zentriert} & \textbf{Rechtsbündig} \\
  1 & Zeile 1 & Zeile 1 & Zeile 1 \\
  2 & Zeile 2 & Zeile 2 & Zeile 2 \\
  3 & Zeile 3 & Zeile 3 & Zeile 3 \\
\end{tabular}
\end{center}
\end{frame}

\subsection{Mehrspaltige Folien}

\begin{frame}
\frametitle{Mehrspaltige Folien}
\begin{columns}
        \column{.45\textwidth}
                Mehrspaltige Folien können einfach mit \texttt{columns} realisiert werden
        \column{.45\textwidth}
                \begin{enumerate}
                \item Ein Eintrag
                \item Noch ein Eintrag
                \item Ein weiterer Eintrag
                \end{enumerate}
\end{columns}
\end{frame}

\subsection{Quellen}

\begin{frame}[fragile]
\frametitle{Quellen}
\begin{itemize}
\item \verb!http://www2.informatik.hu-berlin.de/~mischulz/beamer.html!
\item {\small \verb!http://www.physik.uni-freiburg.de/~tooleh/latex_beamerkurs.pdf!}
\item Google findet sehr viele hilfreiche Links zum Thema \textbf{LaTeX Beamer}
\end{itemize}
\end{frame}

\end{document}
